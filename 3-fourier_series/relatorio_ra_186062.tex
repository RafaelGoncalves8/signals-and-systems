\documentclass{article}
\setlength\parindent{24pt}
\usepackage[margin=1in]{geometry}
\usepackage{indentfirst}
\usepackage{amsmath}
\usepackage{float}
\usepackage[utf8]{inputenc}

\begin{document}

\title{EA614 - Análise de Sinais \\
\large{EFC3 - Série de Fourier}}
\author{Rafael Gonçalves (186062)}
\date{\today}

\maketitle

\section{Parte Computacional}

a)

No caso de $x(t)$ temos:

\begin{equation}
a_{k} = \frac{1}{T}\int_{\frac{-T}{2}}^{\frac{T}{2}}x(t)e^{-jk\omega_{0}t}dt = \frac{1}{T}\int_{\frac{-T}{2}}^{\frac{T}{2}}\frac{2}{T}te^{-jk\omega_{0}t}dt
\end{equation}

$T = 4s$ então podemos escrever como:

\begin{equation}
    a_k = \frac{2}{T^{2}}\int_{-2}^{2}te^{-jk\omega_{0}t}dt
\end{equation}

Para $k = 0$:

\begin{equation}
    a_0 = \frac{2}{T^{2}}\int_{-2}^{2}te^{0}dt = \frac{2}{T^{2}}\int_{-2}^{2}tdt 
\end{equation}

\begin{equation}
    a_0 = 0
\end{equation}

Para $k \neq 0$ podemos resolver por partes:

\begin{center}
    $\int u dv = u v - \int v du$\break

    $u = t ; \quad  du = dt$\break

   $v = \frac{-e^{jk\omega_0t}}{jk\omega_0}; \quad dv = e^{jk\omega_0t}dt$
\end{center}

Então:

\begin{equation}
    \int te^{-jk\omega_{0}t}dt = -t \frac{e^{jk\omega_0t}}{jk\omega_0} + \int \frac{e^{jk\omega_0t}}{jk\omega_0} dt 
\end{equation}

\begin{equation}
    a_k = \frac{2}{T^2} \left ( \left [ - t\frac{e^{jk\omega_0t}}{jk\omega_0} \right ]_{-2}^2 + \int_{-2}^2 \frac{e^{jk\omega_0t}}{jk\omega_0}dt \right )= \frac{2}{T^2} \left [ \frac{e^{jk\omega_0t}}{jk\omega_0} \left ( t + \frac{1}{jk\omega_0} \right ) \right ] _{-2}^2
\end{equation}

\begin{equation}
    a_k = \frac{2}{T^2} \left [ \frac{-2}{jk\omega_0} \left ( e^{jk\omega_0 2} + e^{-jk\omega_0 2} \right ) + \frac{-1}{k^2 \omega_0 ^2} \left ( e^{jk\omega_0 2} - e^{-jk\omega_0 2} \right ) \right ]
\end{equation}

Usando a fórmula de Euler temos:

\begin{center}
    $\omega_0 = \frac{2\pi}{T}$ e $T = 4s$
\end{center}

\begin{equation}
    a_k = \frac{2}{4^2} \left [ \frac{-8}{jk\pi}\cos(\pi k) - \frac{4j}{k^2 \pi^2}\sin(\pi k) \right ]
\end{equation}

\begin{equation}
    ak = \frac{2j}{k \pi} \cos(k \pi)
\end{equation}

\break\vfill



b)

\end{document}
